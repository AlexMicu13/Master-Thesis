\documentclass[10pt,a4paper]{report}

\usepackage[utf8]{inputenc} % pentru suport diacritice
\usepackage{courier}
% \usepackage[romanian]{babel} % setări pentru limba română


\usepackage[top=3.5cm, bottom=3.5cm, left=3.5cm, right=3cm]{geometry} % dimensiuni pagină și margini
\usepackage{graphicx} % support the \includegraphics command and options

% formatting sections and subsections
\usepackage{textcase}
\usepackage[titletoc, title]{appendix}
\usepackage{titlesec}
\titleformat{\chapter}{\large\bfseries\MakeUppercase}{\thechapter}{2ex}{}[\vspace*{-1.5cm}]
\titleformat*{\section}{\large\bfseries}
\titleformat*{\subsection}{\large\bfseries}
\titleformat*{\subsubsection}{\large\bfseries}

\usepackage{chngcntr}
\counterwithout{figure}{chapter} % no chapter number in figure labels
\counterwithout{table}{chapter} % no chapter number in table labels
\counterwithout{equation}{chapter} % no chapter number in equation labels

\usepackage{booktabs} % for much better looking tables
\usepackage{url} % Useful for inserting web links nicely
\usepackage[bookmarks,unicode,hidelinks]{hyperref}

\usepackage{array} % for better arrays (eg matrices) in maths
\usepackage{paralist} % very flexible & customisable lists (eg. enumerate/itemize, etc.)
\usepackage{verbatim} % adds environment for commenting out blocks of text & for better verbatim
\usepackage{subfig} % make it possible to include more than one captioned figure/table in a single float
\usepackage{enumitem}
\setlist{noitemsep}

% packages used for citations
\usepackage[backend=biber,style=ieee]{biblatex}
\addbibresource{sources.bib}

\usepackage{xcolor}
\usepackage{listings}
\lstset{language=Python,basicstyle={\bfseries \color{black}},keywordstyle={\bfseries \color{blue}}}

%%% HEADERS & FOOTERS
\usepackage{fancyhdr}
\pagestyle{empty}
\renewcommand{\headrulewidth}{0pt}
\renewcommand{\footrulewidth}{0pt}
\lhead{}\chead{}\rhead{}
\lfoot{}\cfoot{\thepage}\rfoot{}

\newcommand{\HeaderLineSpace}{-0.25cm}

\newcommand{\UniTextEN}{National University of Science and Technology\\ POLITEHNICA Bucharest\\
Faculty of Automatic Control and Computers\\
Computer Science and Engineering Department\\}
\newcommand{\DiplomaEN}{MASTER THESIS S1 REPORT}
\newcommand{\AdvisorEN}{Thesis advisor:}
\newcommand{\BucEN}{BUCHAREST}

\newcommand{\UniTextRO}{Universitatea Națională de Știință și Tehnologie\\ POLITEHNICA din București \\
Facultatea de Automatică și Calculatoare\\
Departamentul Calculatoare\\}
\newcommand{\DiplomaRO}{RAPORT S1 DISERTAȚIE}
\newcommand{\AdvisorRO}{Coordonator științific:}
\newcommand{\BucRO}{BUCUREȘTI}

\newcommand{\frontPage}[6]{
\begin{titlepage}
\begin{center}
{\Large #1}  % header (university, faculty, department)
\vspace{50pt}
\begin{tabular}{cc}
\includegraphics[height=3cm]{pics/UNST Poli.png} & 
\includegraphics[height=3cm]{pics/sigla_cs.png}
\end{tabular}

\vspace{105pt}
{\Huge #2}\\                             % diploma project text
\vspace{40pt}
{\Large #3}\\ \vspace{0pt}               % project title
\vspace{40pt}
{\LARGE \Name}\\                         % student name
\end{center}
\vspace{60pt}
\begin{tabular*}{\textwidth}{@{\extracolsep{\fill}}p{6cm}r}
&{\large\textbf{#5}}\vspace{10pt}\\      % scientific advisor
&{\large \AdvisorA}
\end{tabular*}
\vspace{20pt}
\begin{center}
{\large\textbf{#6}}\\
\vspace{0pt}
{\normalsize \Year}
\end{center}
\end{titlepage}
}

\newcommand{\frontPageEN}{\frontPage{\UniTextEN}{\DiplomaEN}{\ProjectTitleEN}{\ProjectSubtitleEN}{\AdvisorEN}{\BucEN}}
\newcommand{\frontPageRO}{\frontPage{\UniTextRO}{\DiplomaRO}{\ProjectTitleRO}{\ProjectSubtitleRO}{\AdvisorRO}{\BucRO}}


\linespread{1.15}
\setlength\parindent{0pt}
\setlength\parskip{.28cm}

%% Abstract macro
\newcommand{\AbstractPage}{
\begin{titlepage}

\thispagestyle{fancy}\setcounter{page}{2}
\clearpage\phantomsection\addcontentsline{toc}{chapter}{Abstract}

\textbf{\large ABSTRACT}\par
\AbstractEN \vfill
\textbf{\large SINOPSIS}\par
\AbstractRO\par\vfill
\end{titlepage}
}


%%%%%%%%%%%%%%%%%%%%%%%%%%%%%%%%%%%%%%%%%%%%%%%%%%
%%
%%          End of template definitions
%%
%%%%%%%%%%%%%%%%%%%%%%%%%%%%%%%%%%%%%%%%%%%%%%%%%%

\newcommand{\ProjectTitleRO}{Comunicație oportunistă în orașe inteligente}
\newcommand{\ProjectTitleEN}{Opportunistic Communication in Smart Cities}
\newcommand{\Name}{Alexandru Micu}
\newcommand{\AdvisorA}{Conf. dr. ing. Radu-Ioan Ciobanu}
\newcommand{\Year}{2025}

% Setări document
\title{Proiect de diplomă}
\author{\Name}
\date{\Year}

%%
%%   Câmpurile aferente rezumatului
%%
\newcommand{\AbstractEN}{
Opportunistic communication in smart cities offers a cost-effective alternative
to traditional cellular networks by leveraging existing infrastructure and
placing edge entities. With the number of connected IoT devices projected to
reach 40 billion by 2030, and 5G networks facing scalability challenges due to
bandwidth limitations and deployment costs, this approach addresses critical
gaps in urban digital transformation.
\vspace{1\baselineskip} \\
The placement of edge nodes can be influenced by various factors, including the
number of users or data sources that need to be served, the types of applications
being supported, and the available network infrastructure. Effective edge node
placement can lead to significant improvements in the performance and efficiency
of applications and services, making it a critical aspect of modern network
design and management.
}

\newcommand{\AbstractRO}{
Comunicarea oportunistă în orașele inteligente oferă o alternativă rentabilă la
rețelele celulare tradiționale, prin utilizarea infrastructurii existente și
plasarea entităților edge (la marginea rețelei). Cu un număr de dispozitive IoT
conectate proiectat să ajungă la 40 de miliarde până în 2030, și cu rețelele 5G
care se confruntă cu provocări de scalabilitate din cauza limitărilor de lățime
de bandă și a costurilor de implementare, această abordare abordează lacune
critice în transformarea digitală urbană.
\vspace{1\baselineskip} \\
Amplasarea nodurilor edge poate fi influențată de diverși factori,
inclusiv numărul de utilizatori sau surse de date care trebuie deservite,
tipurile de aplicații care sunt suportate și infrastructura de rețea disponibilă.
Amplasarea eficientă a nodurilor de margine poate duce la îmbunătățiri semnificative
în performanța și eficiența aplicațiilor și serviciilor, ceea ce o face un aspect
critic al proiectării și gestionării rețelelor moderne.

\vspace{3\baselineskip}
\textbf{Keywords}: Edge Computing, Sustainable Costs, Reduced Latency
}


\begin{document}
\pagenumbering{roman}
\frontPageEN
\frontPageRO
\begingroup
\linespread{1}
\tableofcontents
% \let\clearpage\relax
\endgroup

% Keep the list of figures and the list of tables on the same page.
% \pagestyle{fancy}
% \listoffigures\addcontentsline{toc}{chapter}{List of figures}
% \begingroup
% \let\clearpage\relax
% \listoftables\addcontentsline{toc}{chapter}{List of tables}
% \endgroup

% Include the abstract page. See this definition if page numbers are wrong.
\AbstractPage

% poate fi comentata sau stearsa
% \ThanksPage

% Start the normal page numbering.
\pagenumbering{arabic}

% Textul licentei incepe de aici
\chapter{Introduction}\pagestyle{fancy}

\section{Context}

The global smart city market is projected to reach \$6.4 trillion by 2030, driven
by IoT deployments for infrastructure monitoring, traffic management, and energy
efficiency. However, cellular connectivity costs remain prohibitive—mid-sized
cities like Louisville, Kentucky, spend up to \$2.3M annually on cellular plans
for IoT sensors. While 5G RedCap offers reduced-capability IoT devices
(50 Mbps upload, $<$100 ms latency), its deployment faces urban coverage gaps and
energy inefficiencies, particularly for non-urgent data like waste management or
air quality metrics \cite{GSMA2024} \cite{Madamori2021}.\\
Opportunistic D2D networks mitigate these costs by repurposing mobile entities as
data relays. For instance, Chapel Hill's public buses have been used to relay traffic
sensor data to edge nodes at transit hubs with a latency reduction of over 20
minutes due to their predictable routes (85\% predictability) \cite{Madamori2021}. Similarly, Barcelona's
garbage trucks have reduced cellular costs by 62\% by collecting bin fill-level
data via IEEE 802.11p protocols with a store-carry-forward delay tolerance of $\leq$ 4 hours \cite{Sinaeepourfard_2016}.\\
Citizen participation also plays a crucial role; Fargo has incentivized smartphone
users to share bandwidth for environmental monitoring projects using opt-in
blockchain frameworks like Ethereum's zero-knowledge proofs for GDPR compliance.
This approach achieved a significant offload of cellular traffic during peak
hours—up to 40\% \cite{Gandhi_2023}.

\subsection{Technical Challenges}
Traditional centrality metrics such as betweenness fail in transient networks because
they do not account for dynamic mobility patterns during rush hours or unexpected
events like road closures or natural disasters. For example, Ahmedabad's bus network
experienced a notable increase in data loss when static edge nodes were used during
monsoon-induced route detours \cite{Gandhi_2023}.\\
Hybrid network orchestration is another challenge; while 5G RedCap improves grid
reliability through "hard slicing" techniques suitable for critical services such
as solar PV systems (e.g., China's high-voltage distribution networks), integrating
opportunistic links requires sophisticated software-defined networking (SDN)
solutions that can dynamically route traffic based on real-time congestion levels \cite{GSMA2024} \cite{Liu_2024}.\pagebreak \\
Scalability versus cost remains a significant trade-off; deploying one edge node
per forty IoT devices in Gandhinagar achieved nearly full coverage at an annual
cost of approximately \$18K—a reduction of about 77\% compared to traditional
cellular-only models according to simulations conducted using the ONE Simulator tool \cite{Gandhi_2023}.

\subsection{Emerging Solutions}
Blockchain technology is also being integrated into these systems primarily through
platforms like Hyperledger Fabric which was tested successfully in Skudai Malaysia
where it resolved 92\% of GDPR compliance issues related to anonymizing crowdsourced
air quality data from citizen-participatory networks \cite{Alasbali_2022}.\\
Lastly disaster resilience strategies combining satellite backhaul with VANETs
(Vehicular Ad-Hoc Networks) have been implemented effectively; Japan's "Never Die Network"
maintained 68\% connectivity even during earthquakes using cars as ad-hoc edge nodes
demonstrating potential applications beyond urban settings into emergency response
scenarios globally \cite{Andersson_2014}.

\section{Problem Statement}

The integration of vast numbers of IoT devices into smart city environments
presents a complex challenge, particularly concerning the limitations of
traditional cellular networks in terms of scalability, cost-effectiveness, and
energy efficiency. While emerging technologies like 5G RedCap and opportunistic
communication networks (OppNets) offer promising partial solutions, their
widespread adoption is hindered by several key unresolved issues.\\
Firstly, many existing network deployments rely on static edge node placement
strategies, which are ill-suited for the dynamic conditions of real-world urban
environments. Techniques such as centrality-based node placement assume fixed
network topologies. These assumptions break down when facing disruptions like
traffic detours or adverse weather conditions. For example, a study of Ahmedabad's
bus network found that the use of static edge nodes led to a significant 22\%
data loss during monsoon-induced route deviations \cite{Gandhi_2023}.
This underscores the critical need for dynamic adaptation to maintain reliable
connectivity.\\
Secondly, the architecture of hybrid networks, which combine 5G RedCap for
time-sensitive services with OppNets for less critical data, lacks standardized
protocols to effectively balance cost and quality of service (QoS). This absence
of clear guidelines makes it challenging to optimally manage network resources
and ensure the reliable delivery of data across different service types.\\
Thirdly, current approaches often prioritize network coverage and latency while
neglecting environmental impacts. The deployment of energy-intensive edge nodes
in urban areas could potentially undermine the cost savings achieved by OppNets,
if this is coupled with a higher carbon footprint. Therefore, there is a critical
need for research that addresses this trade-off.\\
Finally, opportunistic networks need robust failover mechanisms to maintain
operation during disaster scenarios. The limitations of current approaches were
highlighted during a disaster scenario when Japan's VANET-based "Never Die Network"
maintained only 68\% connectivity during earthquakes \cite{Andersson_2014}. This
exposed gaps in resilience planning, indicating the need for better designs.

\pagebreak

\section{Objectives}
This thesis aims to tackle key challenges in the implementation of opportunistic
communication networks for smart city applications. By focusing on cost-effective
and resilient solutions for municipal infrastructure, this research seeks to
provide practical guidance for future smart city deployments. The core objectives
are as follows.

\subsection{Algorithmic Edge Node Placement for Dynamic Networks}

The primary goal here is to design and evaluate deterministic algorithms that
can strategically position edge nodes within opportunistic networks, particularly
those leveraging public transit systems. This will involve developing practical
methods, such as greedy or graph-based algorithms, capable of adapting to the
constantly changing conditions typical of urban environments.\\
To achieve this, the algorithms will be assessed using real-world transit datasets.
This real-world data will provide a foundation for simulating various network
conditions and comparing the performance of the newly developed algorithms against
traditional, static centrality metrics. The ultimate aim is to create solutions
that not only minimize latency but also keep deployment costs manageable.\\
\subsection{Enhancing Resilience in the Face of Disasters}

A significant aspect of this thesis involves exploring how to make opportunistic
networks robust enough to withstand infrastructure disruptions. To do this, I
intend to thoroughly analyze how local storage capabilities on bus nodes can
ensure data remains accessible, even during disaster scenarios. The plan is to
develop protocols that prioritize the transmission of essential data related to
critical infrastructure services over less urgent traffic during emergencies. By
focusing on these targeted strategies, I intend to provide a pathway toward
reducing cellular dependency by up to 50\% without compromising the responsiveness
of essential city services.\\

\subsection{Optimized Network Configurations for Municipal Infrastructure:}
The thesis will also focus on optimizing opportunistic network architectures
for specific urban infrastructure monitoring applications. This involves modelling
different network setups tailored to needs such as waste management. To evaluate
the effectiveness of these setups, performance metrics like coverage area and
packet delivery ratio will be compared using simulation tools such as ONE Simulator.
The aim is to achieve at least 90\% coverage with configurations that are
specifically designed to support efficient and cost-effective monitoring of
municipal services.\\
\subsection{Integrating Sustainability into Cost-Effective Models:}
Finally, this thesis will investigate how sustainability metrics can be integrated
into the economic analysis of opportunistic networks. This includes quantifying
the carbon footprint associated with deploying solar-powered edge nodes and nodes
built on existing infrastructure. The aim here is to reduce infrastructure costs
by 25\% compared to traditional cellular deployments while maintaining at least
90\% network coverage and prioritizing environmentally responsible design choices.


%---------------------------
\chapter{State of the art}

Opportunistic networks (OppNets) have emerged as a cost-effective alternative to
traditional cellular networks for IoT deployments in smart cities. By leveraging
existing infrastructure like public transit vehicles and citizen devices as data
relays, OppNets reduce reliance on expensive cellular connectivity. This section
highlights recent advancements in edge node placement strategies and hybrid network
architectures.

\section{Edge Node Placement Strategies}
Effective edge node placement is pivotal for achieving optimal performance within
Opportunistic Networks (OppNets). In particular, placement significantly influences
both the minimization of latency and the maximization of network coverage.
Conventional centrality metrics, such as betweenness centrality, often fall
short in these dynamic contexts because they do not adequately account for the
constantly changing patterns of mobility exhibited by network participants. This
limitation has motivated researchers to investigate algorithmic solutions that
can more dynamically adapt to these changes and maintain network efficiency.\\
One notable approach has been the formulation of edge node selection as an
Influence Maximization problem, resulting in the Minimal Delivery Delay (MDD)
algorithm. This algorithm is explicitly designed to select optimal edge nodes
based on datasets derived from real-world transit systems, such as those found
in cities like Chapel Hill and Louisville \cite{LatencyDefined2021}. Notably,
this approach has been shown to outperform traditional centrality metrics,
achieving a significant reduction in end-to-end latency.\\
Another strategy that has gained attention is the Improved Snake Optimization
(ISO) algorithm. The ISO algorithm functions by dynamically adjusting server
locations in response to predicted traffic flows. This dynamic adaptation
enables the algorithm to reduce infrastructure deployment costs by approximately
30\% when compared to static, non-adaptive deployment strategies \cite{Liu_2024}.

\section{Hybrid Network Architectures}
The hybridization of 5G Reduced Capability (RedCap) technology with Opportunistic
Networks (OppNets) represents a balanced and promising approach for smart city
deployments, integrating the strengths of both technologies. This hybrid model
seeks to leverage the cost efficiencies provided by OppNets while capitalizing
on the guaranteed Quality of Service (QoS) offered by 5G RedCap.\\
5G RedCap case studies have underscored its suitability for critical services,
such as smart grids, and provide evidence of its integration with opportunistic
links for more delay-tolerant IoT applications. For instance, China's
high-voltage distribution networks are deploying hard slicing techniques through
5G RedCap that provide robust grid reliability \cite{GSMA2024}.\\
In a different context, the city of Barcelona's innovative waste management system
leveraged opportunistic communication by equipping its fleet of garbage trucks
with IEEE 802.11p protocols. This enabled the trucks to collect data from smart
bins using store-carry-forward mechanisms, leading to a substantial cellular
cost reduction of sixty-two percent \cite{Sinaeepourfard_2016}.

\section{Disaster Resilience Implementations}
The capacity to maintain continuous operation during disaster scenarios stands as
a critical requirement for any robust smart city infrastructure. The effects of
disasters can severely compromise communication networks, underscoring the need
for innovative solutions that ensure connectivity is preserved. A noteworthy
example of such a solution is Japan's "Never Die Network," designed with disaster
resilience as a central tenet. This network strategically combines satellite
backhaul with vehicular ad-hoc networks (VANETs), leveraging vehicles as
impromptu edge nodes to maintain functionality during emergencies such as
earthquakes. Andersson and Kafle's (2014) research highlights the success of
this architecture, reporting that it could maintain approximately 68\%
connectivity even under severe disruption \cite{Andersson_2014}. The ability to
maintain a degree of connectivity under the most challenging circumstances
highlights the significance of the Never Die Network in the implementation
and design of emergency communication networks.

\begin{comment}
%---------------------------
\chapter{Method}

\section{Corpus}

\subsection{Descriptives}

\section{(Neural) Architecture}

\section{Performance Metrics}


%---------------------------
\chapter{Results}


%---------------------------
\chapter{Discussion}

\section{Performance Comparison}

\section{Limitations}


%---------------------------
\chapter{Conclusions and Future Work}
\end{comment}

% Bibliography / References.
\renewcommand\bibname{References}
\clearpage\phantomsection\addcontentsline{toc}{chapter}{\bibname}
\printbibliography

\end{document}

\begin{comment}
For figures:
\begin{figure}[h]
    \centering
    \includegraphics[width=0.25\textwidth]{mesh}
    \caption{a nice plot}
    \label{fig:mesh1}
\end{figure}
\end{comment}