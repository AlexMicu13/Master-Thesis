\documentclass[12pt,a4paper]{report}

\usepackage[utf8]{inputenc} % pentru suport diacritice
\usepackage{lmodern}
% \usepackage[romanian]{babel} % setări pentru limba română
\renewcommand\familydefault{\sfdefault} % sans serif

\usepackage[margin=2.54cm]{geometry} % dimensiuni pagină și margini
\usepackage{graphicx} % support the \includegraphics command and options

% formatting sections and subsections
\usepackage{textcase}
\usepackage[titletoc, title]{appendix}
\usepackage{titlesec}
\titleformat{\chapter}{\large\bfseries\MakeUppercase}{\thechapter}{2ex}{}[\vspace*{-1.5cm}]
\titleformat*{\section}{\large\bfseries}
\titleformat*{\subsection}{\large\bfseries}
\titleformat*{\subsubsection}{\large\bfseries}

\usepackage{chngcntr}
\counterwithout{figure}{chapter} % no chapter number in figure labels
\counterwithout{table}{chapter} % no chapter number in table labels
\counterwithout{equation}{chapter} % no chapter number in equation labels

\usepackage{booktabs} % for much better looking tables
\usepackage{url} % Useful for inserting web links nicely
\usepackage[bookmarks,unicode,hidelinks]{hyperref}

\usepackage{array} % for better arrays (eg matrices) in maths
\usepackage{paralist} % very flexible & customisable lists (eg. enumerate/itemize, etc.)
\usepackage{verbatim} % adds environment for commenting out blocks of text & for better verbatim
\usepackage{subfig} % make it possible to include more than one captioned figure/table in a single float
\usepackage{enumitem}
\setlist{noitemsep}

% packages used for citations
\usepackage[backend=biber,style=ieee]{biblatex}
\addbibresource{sources.bib}

\usepackage{xcolor}
\usepackage{listings}
\lstset{language=Python,basicstyle={\bfseries \color{black}},keywordstyle={\bfseries \color{blue}}}

%%% HEADERS & FOOTERS
\usepackage{fancyhdr}
\pagestyle{empty}
\renewcommand{\headrulewidth}{0pt}
\renewcommand{\footrulewidth}{0pt}
\lhead{}\chead{}\rhead{}
\lfoot{}\cfoot{\thepage}\rfoot{}

\newcommand{\HeaderLineSpace}{-0.25cm}

\newcommand{\UniTextEN}{National University of Science and Technology\\ POLITEHNICA Bucharest\\
Faculty of Automatic Control and Computers\\
Computer Science and Engineering Department\\}
\newcommand{\DiplomaEN}{MASTER THESIS RESEARCH REPORT}
\newcommand{\AdvisorEN}{Thesis advisor:}
\newcommand{\BucEN}{BUCHAREST}

\newcommand{\UniTextRO}{Universitatea Națională de Știință și Tehnologie\\ POLITEHNICA din București \\
Facultatea de Automatică și Calculatoare\\
Departamentul Calculatoare\\}
\newcommand{\DiplomaRO}{RAPORT CERCETARE DISERTAȚIE}
\newcommand{\AdvisorRO}{Coordonator științific:}
\newcommand{\BucRO}{BUCUREȘTI}

\newcommand{\frontPage}[6]{
\begin{titlepage}
\begin{center}
{\Large #1}  % header (university, faculty, department)
\vspace{50pt}
\begin{tabular}{ccc}
\includegraphics[height=3cm]{pics/UNST Poli.png} & 
\includegraphics[height=3cm]{pics/A&C.png} & 
\includegraphics[height=3cm]{pics/sigla_cs.png}
\end{tabular}

\vspace{105pt}
{\Huge #2}\\                             % diploma project text
\vspace{40pt}
{\Large #3}\\ \vspace{0pt}               % project title
\vspace{40pt}
{\LARGE \Name}\\                         % student name
\end{center}
\vspace{60pt}
\begin{tabular*}{\textwidth}{@{\extracolsep{\fill}}p{6cm}r}
&{\large\textbf{#5}}\vspace{10pt}\\      % scientific advisor
&{\large \AdvisorA}
\end{tabular*}
\vspace{20pt}
\begin{center}
{\large\textbf{#6}}\\
\vspace{0pt}
{\normalsize \Year}
\end{center}
\end{titlepage}
}

\newcommand{\frontPageEN}{\frontPage{\UniTextEN}{\DiplomaEN}{\ProjectTitleEN}{\ProjectSubtitleEN}{\AdvisorEN}{\BucEN}}
\newcommand{\frontPageRO}{\frontPage{\UniTextRO}{\DiplomaRO}{\ProjectTitleRO}{\ProjectSubtitleRO}{\AdvisorRO}{\BucRO}}


\linespread{1.15}
\setlength\parindent{0pt}
\setlength\parskip{.28cm}

%% Abstract macro
\newcommand{\AbstractPage}{
\begin{titlepage}

\thispagestyle{fancy}\setcounter{page}{5}
\clearpage\phantomsection\addcontentsline{toc}{chapter}{Abstract}

\textbf{\large ABSTRACT}\par
\AbstractEN \vfill
\textbf{\large SINOPSIS}\par
\AbstractRO\par\vfill
\end{titlepage}
}


%%%%%%%%%%%%%%%%%%%%%%%%%%%%%%%%%%%%%%%%%%%%%%%%%%
%%
%%          End of template definitions
%%
%%%%%%%%%%%%%%%%%%%%%%%%%%%%%%%%%%%%%%%%%%%%%%%%%%

\newcommand{\ProjectTitleRO}{Comunicație oportunistă în orașe inteligente}
\newcommand{\ProjectTitleEN}{Opportunistic Communication in Smart Cities}
\newcommand{\Name}{Alexandru Micu}
\newcommand{\AdvisorA}{Conf. dr. ing. Radu Ioan Ciobanu}
\newcommand{\Year}{2025}

% Setări document
\title{Proiect de diplomă}
\author{\Name}
\date{\Year}

%%
%%   Câmpurile aferente rezumatului
%%
\newcommand{\AbstractEN}{
Opportunistic communication in smart cities offers a cost-effective alternative\\
to traditional cellular networks by leveraging existing infrastructure and\\
placing edge entities. With the number of connected IoT devices projected to\\
reach 40 billion by 2030, and 5G networks facing scalability challenges due to\\
bandwidth limitations and deployment costs, this approach addresses critical\\
gaps in urban digital transformation.
\vspace{1\baselineskip} \\
The placement of edge nodes can be influenced by various factors, including the\\
number of users or data sources that need to be served, the types of applications\\
being supported, and the available network infrastructure. Effective edge node\\
placement can lead to significant improvements in the performance and efficiency\\
of applications and services, making it a critical aspect of modern network\\
design and management.
}

\newcommand{\AbstractRO}{
Comunicarea oportunistă în orașele inteligente oferă o alternativă rentabilă la\\
rețelele celulare tradiționale, prin utilizarea infrastructurii existente și\\
plasarea entităților edge (la marginea rețelei). Cu un număr de dispozitive IoT\\
conectate proiectat să ajungă la 40 de miliarde până în 2030, și cu rețelele 5G\\
care se confruntă cu provocări de scalabilitate din cauza limitărilor de lățime\\
de bandă și a costurilor de implementare, această abordare abordează lacune\\
critice în transformarea digitală urbană.
\vspace{1\baselineskip} \\
Amplasarea nodurilor edge poate fi influențată de diverși factori,\\
inclusiv numărul de utilizatori sau surse de date care trebuie deservite,\\
tipurile de aplicații care sunt suportate și infrastructura de rețea disponibilă.\\
Amplasarea eficientă a nodurilor de margine poate duce la îmbunătățiri semnificative\\
în performanța și eficiența aplicațiilor și serviciilor, ceea ce o face un aspect\\
critic al proiectării și gestionării rețelelor moderne.

\vspace{3\baselineskip}
\textbf{Keywords}: Edge Computing, Sustainable Costs, Reduced Latency
}


\begin{document}
\pagenumbering{roman}
\frontPageRO
\frontPageEN
\begingroup
\linespread{1}
\tableofcontents
\endgroup

% % Keep the list of figures and the list of tables on the same page.
% \pagestyle{fancy}
% \listoffigures\addcontentsline{toc}{chapter}{List of figures}
% \begingroup
% \let\clearpage\relax
% \listoftables\addcontentsline{toc}{chapter}{List of tables}
% \endgroup

% Include the abstract page. See this definition if page numbers are wrong.
\AbstractPage

% poate fi comentata sau stearsa
% \ThanksPage

% Start the normal page numbering.
\pagenumbering{arabic}

% Textul licentei incepe de aici
\chapter{Introduction}\pagestyle{fancy}

\section{Context}

The global smart city market is projected to reach \$6.4 trillion by 2030, driven
by IoT deployments for infrastructure monitoring, traffic management, and energy
efficiency. However, cellular connectivity costs remain prohibitive—mid-sized
cities like Louisville, Kentucky, spend up to \$2.3M annually on cellular plans
for IoT sensors. While 5G RedCap offers reduced-capability IoT devices
(50 Mbps upload, $<$100 ms latency), its deployment faces urban coverage gaps and
energy inefficiencies, particularly for non-urgent data like waste management or
air quality metrics \cite{GSMA2024} \cite{Madamori2021}.\\
Opportunistic D2D networks mitigate these costs by repurposing mobile entities as
data relays. For instance, Chapel Hill's public buses have been used to relay traffic
sensor data to edge nodes at transit hubs with a latency reduction of over 20
minutes due to their predictable routes (85\% predictability) \cite{Madamori2021}. Similarly, Barcelona's
garbage trucks have reduced cellular costs by 62\% by collecting bin fill-level
data via IEEE 802.11p protocols with a store-carry-forward delay tolerance of $\leq$ 4 hours \cite{Gandhi_2023}.\\
Citizen participation also plays a crucial role; Fargo has incentivized smartphone
users to share bandwidth for environmental monitoring projects using opt-in
blockchain frameworks like Ethereum's zero-knowledge proofs for GDPR compliance.
This approach achieved a significant offload of cellular traffic during peak
hours—up to 40\% \cite{Gandhi_2023}.

\subsection{Technical Challenges}
Traditional centrality metrics such as betweenness fail in transient networks because
they do not account for dynamic mobility patterns during rush hours or unexpected
events like road closures or natural disasters. For example, Ahmedabad's bus network
experienced a notable increase in data loss when static edge nodes were used during
monsoon-induced route detours \cite{Gandhi_2023}.\\
Hybrid network orchestration is another challenge; while 5G RedCap improves grid
reliability through "hard slicing" techniques suitable for critical services such
as solar PV systems (e.g., China's high-voltage distribution networks), integrating
opportunistic links requires sophisticated software-defined networking (SDN)
solutions that can dynamically route traffic based on real-time congestion levels \cite{GSMA2024} \cite{Liu_2024}.\pagebreak \\
Scalability versus cost remains a significant trade-off; deploying one edge node
per forty IoT devices in Gandhinagar achieved nearly full coverage at an annual
cost of approximately \$18K—a reduction of about 77\% compared to traditional
cellular-only models according to simulations conducted using the ONE Simulator tool \cite{Gandhi_2023}.
\subsection{Emerging Solutions}
Blockchain technology is also being integrated into these systems primarily through
platforms like Hyperledger Fabric which was tested successfully in Skudai Malaysia
where it resolved 92\% of GDPR compliance issues related to anonymizing crowdsourced
air quality data from citizen-participatory networks \cite{Alasbali_2022}.\\
Lastly disaster resilience strategies combining satellite backhaul with VANETs have
been implemented effectively; Japan's "Never Die Network" maintained 68\% connectivity
even during earthquakes using cars as ad-hoc edge nodes demonstrating potential
applications beyond urban settings into emergency response scenarios globally \cite{Andersson_2014}.

\section{Problem Statement}

\section{Objectives}

%---------------------------
\chapter{State of the art}


\begin{comment}
%---------------------------
\chapter{Method}

\section{Corpus}

\subsection{Descriptives}

\section{(Neural) Architecture}

\section{Performance Metrics}


%---------------------------
\chapter{Results}


%---------------------------
\chapter{Discussion}

\section{Performance Comparison}

\section{Limitations}


%---------------------------
\chapter{Conclusions and Future Work}
\end{comment}

% Bibliography / References.
\renewcommand\bibname{References}
\clearpage\phantomsection\addcontentsline{toc}{chapter}{\bibname}
\printbibliography

\end{document}
